\chapter{Existing Platforms}
In order to better understand what instruments people use when interacting with music, it would be better
to look at the already working solutions. The descriptions provided, instead of giving general information,
will be focusing on the social aspects of the platforms.


\section{Streaming Services}
As it can be seen from the table (ref to survey table) and further confirmed by
the recent study of the International Federation of the Phonographic Industry\cite{2},
nowadays, the most prevalent way of music consumption
and discovery are the streaming services. There are many existing platforms,
but I suggest we look only at those which are both popular and have unique features:

- Spotify.
One of the most prominent social features on Spotify is the 'Spotify Jam'\cite{3}.
It lets people create a collective song queue which is then synchronized among all connected users.
Moreover, volume, the order of songs and other aspects of the playback can be controlled individually.
Another notable tool is the 'Blend' playlists\cite{4}. These are playlists created automatically
between two people, which contain songs matching audio preferences of both users.
Lastly, 'Friend Activity'\cite{5}, which shows what the people you follow are currently listening to,
and 'Listening Parties' that are live chats with limited capacity,
which can be joined for a short time when new music is being released\cite{6,7}.

- VK Music.
As was mentioned previously, this a music service integrated into the VK social network.
Consequently, it is possible to send songs and playlists via private messages and add audio materials to
posts in groups. VK Music also supports algorithmic playlists based on the groups that a user follows.
It analyzes the audio content posted in the specific groups and puts similar songs in the before mentioned playlists.

- SoundCloud.
SoundCloud is one of the few platforms which lets its users leave comments and reactions on songs and playlists\cite{8,9}.
In addition, each user has a personal feed, consisting of his personal uploads and
reposts of other's content\cite{10}, which is visible when visiting his profile.

- Bandcamp.
Every Bandcamp user has a profile with 4 tabs - 'collection', 'wishlist', 'followers' and 'following'.
By far the most interesting feature is under the `following` tab - it is possible to list the genres
which you would like to appear to other users looking at your profile. The 'whishlist' tab is also worth mentioning,
as Bandcamp's model combines streaming and traditional buying of separate releases, in this tab the user is able to
show what he is looking forward to listening in the future.


\section{Forums}
A lot of different resources are gathered under this umbrella term
TODO


\section{Other Platforms}
TODO