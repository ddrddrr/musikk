\chapter{Existing Platforms}
In order to better understand what instruments people use when interacting with music,
it is useful to examine existing solutions. Additionally, this research is going to provide an insight
into the possible features to be implemented in the resulting application.\\
Firstly, streaming services are discussed, as the main aim of the thesis is to create a system for music streaming.
Then other music-related portals are listed, as they offer alternative possibilities to engage with audio content,
other than its direct consumption.
And lastly,

% TODO: add pictures for examples without links?


\section{Streaming Services}
% TODO: ref to survey table
As can be seen in the table~\ref{tab:where_listen} and further confirmed by
the recent study of the International Federation of the Phonographic Industry\cite{music_stats_2024},
nowadays, the most prevalent way of music consumption
and discovery is through the streaming services. There are many options available,
but only those that are both popular and have unique elements will be considered in this thesis.
The descriptions, rather than offering general portrayal of each platform,
will focus on the service-specific features that include social elements.

\subsection{Spotify}
One of the most prominent social features on Spotify is the 'Spotify Jam'\cite{spotify_jam}.
It lets people create a collective song queue which is then synchronized among all connected users.
Moreover, volume, the order of songs and other aspects of the playback can be controlled individually.
Another notable tool is the 'Blend' playlists\cite{spotify_recs}. These are playlists created automatically
between two people, which contain songs matching audio preferences of both users.
Lastly, 'Friend Activity'\cite{spotify_friend_activ}, which shows what the people you follow are currently listening to,
and 'Listening Parties' which are live chats with a limited capacity,
that can be joined for a short time when new music is being released\cite{spotify_party_1,spotify_party_2}.

\subsection{VK Music}
As mentioned previously, this is a music service integrated into the VK social network.
Consequently, it is possible to send songs and playlists via private messages and add audio materials to
posts in groups. VK Music also provides an 'Updates' tab, which is populated by the audio materials that
any followed user or group has added to their 'Liked' feed.

\subsection{SoundCloud}
SoundCloud is one of the few platforms which lets its users leave comments
and reactions on songs and playlists\cite{sc_comments,sc_reactions}.
In addition, each user has a feed consisting of personal uploads and
reposts of other's content\cite{sc_reposts}, which is visible when visiting their profile.

\subsection{Bandcamp}
Every Bandcamp user has a profile with 4 tabs: 'collection', 'wishlist', 'followers' and 'following'.
By far the most interesting feature is under the `following` tab -- users can subscribe to available genres and then
specify the ones that they want to showcase to others viewing their profile.
The ‘wishlist’ tab is also noteworthy,
as Bandcamp’s model blends streaming with the traditional purchases of individual releases, and
in this tab the user is able to show what he is looking forward to listening in the future.
\\\\
The table~\ref{tab:social_features} summarizes the comparison between the features mentioned above.\\
\textbf{Difficulty} suggests the possible complexity of implementation.\\
\textbf{Mode} assumes the typical context in which the feature operates.\\
\textbf{Social Interaction Level} indicates the level of user involvement, when using the feature.\\

%TODO: format table
\begin{table}[ht]
    \centering
    \caption{Comparison of Service‐Specific Social Features}
    \label{tab:social_features}
    \begin{tabular}{|l|l|c|c|c|}
        \hline
        \textbf{Feature}        & \textbf{Service} & \textbf{Difficulty} & \textbf{Mode}    & \textbf{Social Interaction Level} \\
        \hline\hline
        Spotify Jam             & Spotify          & High                & Group            & High                              \\
        \hline
        Blend playlists         & Spotify          & Medium              & Pair             & Low                               \\
        \hline
        Friend Activity         & Spotify          & Low                 & Pair             & Medium                            \\
        \hline
        Listening Parties       & Spotify          & Medium              & Group            & Medium                            \\
        \hline
        Music sending via chats & VK Music         & Medium              & Individual/Group & High                              \\
        \hline
        Embedding into posts    & VK Music         & Medium              & Individual/Group & Medium                            \\
        \hline
        Updates tab             & VK Music         & Low                 & Individual       & High                              \\
        \hline
        Comments                & SoundCloud       & Low                 & Group            & High                              \\
        \hline
        Reposts feed            & SoundCloud       & Low                 & Individual       & Medium                            \\
        \hline
        Genre following         & Bandcamp         & Medium              & Individual       & Low                               \\
        \hline
        Wishlist                & Bandcamp         & Low                 & Individual       & Low                               \\
        \hline
    \end{tabular}
\end{table}


\section{Forums, Blogs and other Music Media}
A lot of different resources are gathered under these umbrella terms -- from professional music review websites to
amateur, personal pages. Again, only widely-used services that offer something remarkable are listed.

\subsection{Rate Your Music}

\begin{quote}
    `Rate Your Music is one of the largest music databases online. It is an incredible tool that
    can help you find and learn about new music to listen to.'\cite{ryt}.\\
\end{quote}
This website is one of the biggest platforms with user-created content.
Every member is able to write reviews and set ratings for music releases,
contribute to the extensive 'wiki' consisting of genres, thematic lists and charts,
and connect to other members of the forum. However, the content is still moderated -- any post has
to be properly formatted and abide to the guidelines in order to appear on the website.

\subsection{2step.ru}
2step.ru is one of the better representatives of 'old school' forums that is active to this day.
This is a country-specific resource, and as a consequence of that,
on top of providing the regular music and forum components, there are elements which are usually missing
from bigger portals. For example, a section about upcoming parties,
a page dedicated to local and upcoming DJs, and an ongoing list of recorded radio shows.\cite{2step}

% TODO: rewrite

\subsection{last.fm}
Last.fm is a music discovery and social networking service built around its ‘scrobbling’ feature,
which automatically records every track played through connected players to the user’s profile,
creating an exhaustive listening history\cite{lastfm}.
Based on this data, Last.fm generates personalized recommendations and charts,
while tag-based navigation enables exploration of genres and user-curated collections\cite{lastfm_tags}.
Social interaction is fostered through friend lists, public groups for discussion and the ability to comment
on artist and track pages. Additionally, an `Events' section aggregates concert listings and allows users to
indicate that they are interested or confirm that they are going and to add comments under each event post,
bridging online listening activity with real-world live music experiences\cite{lastfm_events}.

In summary, these platforms illustrate the range of community-driven and data-driven approaches to music engagement
beyond pure streaming.
They highlight how social interaction and user-generated content can enrich interactions with music-related content.


\section{Other Platforms}
% TODO: add ref to survey
Another memorable outcome of the survey is the high percentage of music discovery on non-music specific platforms.
Instagram, YouTube, TikTok and other services that are able to host user-created content can have a major effect on
listening habits. In recent years, music has been more deeply integrated into these platforms.
For instance, Instagram introduced a dedicated audio tool to seamlessly embed sounds and songs into ‘Reels’\cite{inst_audio},
while YouTube automatically detects songs used in videos and adds them to the video description.


\section*{Summary}
As it can be clearly seen, there are numerous instruments available across the Internet.
However, services that focus on the actual audio delivery often lack features stretching beyond that,
forcing users to use multiple platforms in order to meet their needs. This project aims to bridge that gap,
particularly by enhancing the potential for social interaction among users.
% TODO: say smth about the table and in general, why we did that, what we learned
% TODO: add a ref to the visitor counts

