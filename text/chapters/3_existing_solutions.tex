\chapter{Existing Platforms}
In order to better understand what instruments people use when interacting with music,
it’s useful to examine existing solutions.


% TODO: add pictures for examples without links?


\section{Streaming Services}
% TODO: ref to survey table
As it can be seen from the table~\ref{tab:where_listen} and further confirmed by
the recent study of the International Federation of the Phonographic Industry\cite{music_stats_2024},
nowadays, the most prevalent way of music consumption
and discovery is the streaming services. There are many platforms available,
but only those that are both popular and feature unique elements will be considered.
The descriptions, rather than offering general information, will focus on the social aspects of each platform.

\begin{itemize}
    \item \textbf{Spotify} \\
    One of the most prominent social features on Spotify is the 'Spotify Jam'\cite{spotify_jam}.
    It lets people create a collective song queue which is then synchronized among all connected users.
    Moreover, volume, the order of songs and other aspects of the playback can be controlled individually.
    Another notable tool is the 'Blend' playlists\cite{spotify_recs}. These are playlists created automatically
    between two people, which contain songs matching audio preferences of both users.
    Lastly, 'Friend Activity'\cite{spotify_friend_activ}, which shows what the people you follow are currently listening to,
    and 'Listening Parties' that are live chats with a limited capacity,
    which can be joined for a short time when new music is being released\cite{spotify_party_1,spotify_party_2}.

    \item \textbf{VK Music} \\
    As was mentioned previously, this a music service integrated into the VK social network.
    Consequently, it is possible to send songs and playlists via private messages and add audio materials to
    posts in groups. VK Music also supports algorithmic playlists based on the groups that a user follows.
    It analyzes audio files posted in the specific groups and puts similar songs in those playlists.

    \item \textbf{SoundCloud} \\
    SoundCloud is one of the few platforms which lets its users leave comments
    and reactions on songs and playlists\cite{sc_comments,sc_reactions}.
    In addition, each user has a feed, consisting of his personal uploads and
    reposts of other's content\cite{sc_reposts}, which is visible when visiting his profile.

    \item \textbf{Bandcamp} \\
    Every Bandcamp user has a profile with 4 tabs - 'collection', 'wishlist', 'followers' and 'following'.
    By far the most interesting feature is under the `following` tab - users can subscribe to available genres and then
    specify the ones that they want to showcase to others viewing their profile.
    The ‘wishlist’ tab is also noteworthy,
    as Bandcamp’s model blends streaming with the traditional purchases of individual releases, and
    in this tab the user is able to show what he is looking forward to listening in the future.
\end{itemize}


\section{Forums, Blogs and other Music Media}
A lot of different resources are gathered under these umbrella terms - from professional music review websites to
amateur, personal pages. Again, only widely-used services that offer something remarkable are listed.

\begin{itemize}
    \item \textbf{Rate Your Music} \\
    'Rate Your Music is one of the largest music databases online. It is an incredible tool that
    can help you find and learn about new music to listen to.'\cite{ryt}.\\
    This website is one of the biggest platforms with user-created content.
    Every member is able to write reviews and set ratings for music releases,
    contribute to the extensive 'wiki' consisting of genres, thematic lists and charts,
    and connect to other members of the forum. However, the content is still moderated,
    ensuring that only properly formatted reviews that do not violate guidelines can appear on the website.

    \item \textbf{2step.ru} \\
    One of the better representatives of 'old school' forums that is active to this day.
    This is a country-specific resource, and a consequence of that,
    on top of providing the regular music and forum components, there are elements which are usually missing
    from bigger portals. For example, a section about upcoming parties,
    a page dedicated to local and upcoming DJs, and an ongoing list of recorded radio shows.\cite{2step}

    \item \textbf{The Wire} \\
    The Wire is a long-running magazine with a strong online presence,
    known for its focus on experimental and underground music.
    Apart from conventional articles, reviews, and interviews it hosts podcasts, creates music compilations and curates
    video/photo collections.\cite{thewire}
\end{itemize}


\section{Other Platforms}
Another memorable outcome of the survey is the high percentage of music discovery on non-music specific platforms.
Instagram, YouTube, TikTok and other services that are able to host user-created content can have a major effect on
the listening habits. In recent years, music has been more deeply integrated into these platforms.
For instance, Instagram introduced a dedicated audio tool to seamlessly embed sounds and songs into ‘Reels’\cite{inst_audio},
while YouTube automatically detects songs used in videos and adds them to the video description.


\section*{Summary}
As it can be clearly seen, there are numerous instruments available across the Internet.
However, services offering audio content often lack many commonly used features,
forcing users to use multiple platforms in order to meet their needs. This project aims to bridge that gap,
particularly by enhancing the potential for social interaction among users.


