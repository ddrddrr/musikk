\chapter{Introduction}

In recent years, with the rapid development of the Internet,
music has become an even more integral part of everyday life.
It has never been easier to experience and share music — we have come
a long way from sharing physical media to simply sending a link to a streaming platform of choice.
Consequently, music has integrated even deeper into social interactions between people,
helping them bond and share strong emotional experiences~\cite{1}.

One of the direct consequences of this trend is the fast emergence of numerous music-related platforms.
While some focus on traditional music journalism or statistics, others offer unlimited access to audio content.
Naturally, people have started to discover and engage with music that resonates with them more frequently.

Despite this, it is surprising that none of the major social networking platforms has integrated
a fully-fledged music streaming service into its ecosystem.
There are only two notable exceptions:

\begin{itemize}
    \item \textbf{VK} — a Russian social network that includes a built-in music streaming service.
    \item \textbf{QQ} — a Chinese streaming platform with basic social integration, mostly tailored to content authors.
\end{itemize}

However, both platforms are limited in terms of broader functionality, which will be discussed in later chapters.\\

The goal of this thesis is to design a music-centric platform that supports
collaboration and social interaction around music.

This work is divided into the following \textbf{six} chapters:

\begin{enumerate}
    \item \textbf{Survey Results:}
    Presents the outcomes of a survey illustrating how people consume music, how prevalent it is in social interactions, and why this thesis is relevant.

    \item \textbf{Platform Comparison:}
    Compares existing streaming solutions and explores non-musical platforms that influence people’s audio habits.

    \item \textbf{Specification:}
    Details the implementation plan, overall structure of the application, and key features to be developed.

    \item \textbf{Application Outline:}
    Describes the application's interface and functional outline.

    \item \textbf{Implementation and Design:}
    Explains implementation details and design choices made during development.

    \item \textbf{Conclusion:}
    Summarizes the results and discusses possible improvements.
\end{enumerate}