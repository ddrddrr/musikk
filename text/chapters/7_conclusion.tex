\chapter{Summary and Conclusion}\label{chap:conclusion}
This thesis set out to design and implement a modern music streaming platform
with integrated social features, addressing the observed gap
in user interaction capabilities across existing services. Based on the outcomes of
a conducted survey and analysis of current solutions, the project identified
an existing user interest in socially-enhanced music consumption and sharing.

The platform was implemented using a Django-based backend and a React-based frontend,
chosen for their development efficiency, community support, and extensibility.
The audio delivery is performed using adaptive streaming via DASH and HLS protocols,
ensuring compatibility and quality across devices and browsers.
Opus and AAC codecs were utilized with multiple CBR representations to support adaptive bitrate
switching. Server-Sent Events (SSE) were implemented to provide real-time
updates with low complexity and good reliability. In addition,
the resulting application is synchronized across different instances.

The resulting system supports all basic operations expected from a music streaming platform,
while adding additional social features.
The social interaction capabilities were designed to be lightweight yet functional,
including post and comment systems, real-time friend activity updates.

Deployment was carried out using Docker, PostgreSQL, and a Caddy server,
and the application is now publicly available online.
Key challenges, such as browser inconsistencies with playback state persistence
and device lifecycle handling, were identified, and workarounds
or future improvement paths were proposed.

In conclusion, the project demonstrates that a socially oriented music streaming
platform can be effectively implemented with modern web technologies.
It fulfills the identified requirements, introduces meaningful social
interactions, and presents a flexible foundation for further enhancements,
such as collaborative playlists, advanced discovery algorithms,
or integration with third-party media platforms.
Future development will aim to expand these features,
improve performance, and ensure long-term scalability.


